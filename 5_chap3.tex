\chapter{Практическая часть}
\section{Анализ и структура реализации музыкального плеера}
Проект состоит из одной основной активности и одного основного компонента для интерфейса.

\subsection{Структура проекта}

В проекте определена одна активность — \texttt{MediaPlayerActivity}, которая служит точкой входа и отвечает за запуск пользовательского интерфейса с помощью функции \texttt{setContent}. Внутри вызывается основная компонуемая функция \texttt{MusicLibraryUI}, реализующая весь интерфейс и логику плеера.

\subsection{Функция интерфейса MusicLibraryUI}

Функция \texttt{MusicLibraryUI} выполняет ключевые задачи приложения:

\begin{itemize}
  \item Запрашивает у пользователя разрешение на чтение медиафайлов;
  \item Загружает аудиофайлы из папки \texttt{Music} во внешнем хранилище;
  \item Отображает список доступных треков с помощью компонента \texttt{LazyColumn};
  \item Управляет воспроизведением треков с помощью \texttt{MediaPlayer};
  \item Следит за состоянием плеера и обновляет интерфейс (позиция, длительность и т.д.).
\end{itemize}

\subsection{Запрос разрешений}

Для корректной работы с файловой системой необходимо запросить соответствующие разрешения у пользователя. В зависимости от версии Android используются:

\begin{itemize}
    \item \texttt{READ\_MEDIA\_AUDIO} --- начиная с Android 13 (API 33);
    \item \texttt{READ\_EXTERNAL\_STORAGE} --- для Android 12 и ниже.
\end{itemize}

Разрешения запрашиваются через механизм \texttt{rememberLauncherForActivityResult} из Compose API.

\subsection{Загрузка аудиофайлов}

Файлы с расширениями \texttt{.mp3}, \texttt{.wav}, \texttt{.ogg} загружаются с помощью функции \texttt{loadAudioFiles}, которая ищет их в стандартной папке \texttt{Environment.DIRECTORY\_MUSIC}.

\subsection{Управление воспроизведением}

Воспроизведение реализовано с помощью Android API \texttt{MediaPlayer}. Функция \texttt{playTrack(index)} отвечает за выбор и воспроизведение трека. Также предусмотрены следующие элементы управления:

\begin{itemize}
    \item Переключение на следующий и предыдущий трек;
    \item Кнопка воспроизведения и паузы;
    \item Отображение текущей позиции воспроизведения и полной длительности трека.
\end{itemize}

Состояние обновляется в режиме реального времени с помощью корутин и реактивных состояний Compose.

\subsection{Пользовательский интерфейс}

Интерфейс построен на компонентах Jetpack Compose: \texttt{Column}, \texttt{LazyColumn}, \texttt{ListItem}, \texttt{IconButton}. При выборе трека пользователь видит его название, а также может управлять воспроизведением через удобный набор иконок. Приложение динамически обновляет отображаемую информацию в зависимости от выбранного трека и текущего состояния плеера.

\newpage