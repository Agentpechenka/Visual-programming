\chapter{Теоретическая часть}

\subsection{Jetpack Compose}
Jetpack Compose представляет собой декларативный подход к разработке пользовательских интерфейсов.
В отличие от традиционного XML, в Compose UI описывается c помощью функций на Kotlin \cite{kotlin-doc}.
Это упрощает разработку и позволяет избежать избыточности кода.
Например, элемент интерфейса может быть создан как простая функция:

\begin{lstlisting}
@Composable
fun Greeting(name: String) {
    Text(text = "Hello $name!")
}
\end{lstlisting}


\subsection{MediaPlayer API}
Для воспроизведения аудио в Android используется класс \texttt{MediaPlayer}\cite{android-mediaplayer}. Он предоставляет следующие возможности:
\begin{itemize}
  \item Загрузка локальных и удалённых аудиофайлов;
  \item Поддержка различных форматов (MP3, WAV, OGG и др.);
  \item Управление воспроизведением (пауза, стоп, перемотка);
  \item Работа в фоновом режиме.
\end{itemize}

Пример использования MediaPlayer:

\begin{lstlisting}
val player = MediaPlayer()
player.setDataSource("path/to/file.mp3")
player.prepare()
player.start()
\end{lstlisting}

\subsection{Система разрешений в Android}

Начиная с Android 6.0 (API 23), система безопасности требует запроса разрешений во время выполнения (runtime permissions). Для доступа к файлам, расположенным во внешнем хранилище, необходимо явно запрашивать разрешение у пользователя. Android разделяет разрешения на два типа: обычные (normal) и опасные (dangerous). Чтение медиафайлов относится ко второму типу.

С Android 13 (API 33) были введены новые отдельные разрешения, включая \texttt{READ\_MEDIA\_AUDIO}\cite{android-permissions}, позволяющее приложению читать только аудиофайлы. Это позволяет более точно контролировать доступ приложений к пользовательским данным.

В Jetpack Compose для запроса разрешения используется специальный механизм:

\begin{lstlisting}
val launcher = rememberLauncherForActivityResult(
    ActivityResultContracts.RequestPermission()
) { isGranted ->
    if (isGranted) {
        // Разрешение получено
    }
}
\end{lstlisting}

\subsection{Работа с состоянием в Compose}

Jetpack Compose использует реактивную модель состояния. Это означает, что UI автоматически обновляется при изменении состояния. Для хранения состояния используется API \texttt{remember} и \texttt{mutableStateOf}:

\begin{lstlisting}
var isPlaying by remember { mutableStateOf(false) }

Button(onClick = { isPlaying = !isPlaying }) {
    Text(if (isPlaying) "Pause" else "Play")
}
\end{lstlisting}

Состояния в Compose позволяют создавать динамические и отзывчивые интерфейсы без необходимости ручного обновления UI-элементов.

\subsection{Обработка жизненного цикла и ресурсов}

При работе с мультимедиа важно учитывать жизненный цикл активности. Объекты, такие как \texttt{MediaPlayer}, требуют ручного освобождения ресурсов во избежание утечек памяти. Compose предоставляет функцию \texttt{DisposableEffect}, которая позволяет управлять ресурсами:

\begin{lstlisting}
DisposableEffect(Unit) {
    onDispose {
        player.release()
    }
}
\end{lstlisting}

Это особенно полезно для компонентов, которые не управляются фреймворком напрямую, как в случае с Java- или Android-классами.

\subsection{Архитектурные подходы: MVVM}

Несмотря на то, что демонстрационное приложение реализовано в одном файле, для более масштабных проектов рекомендуется использовать архитектурный паттерн MVVM (Model-View-ViewModel). В нём:

\begin{itemize}
  \item \textbf{Model} — содержит бизнес-логику и доступ к данным (например, загрузка аудиофайлов);
  \item \textbf{ViewModel} — содержит состояние UI и бизнес-логику, обрабатывает события;
  \item \textbf{View} — Jetpack Compose-компоненты, подписанные на данные ViewModel.
\end{itemize}

Jetpack предоставляет библиотеку \texttt{androidx.lifecycle.viewmodel.compose.viewModel}, которая позволяет интегрировать ViewModel в Compose-приложения.

\subsection{Преимущества использования Jetpack Compose}

Jetpack Compose обладает рядом преимуществ по сравнению с традиционным подходом через XML:

\begin{itemize}
  \item Более простая и лаконичная запись интерфейсов;
  \item Возможность полной декларативности и реактивности;
  \item Отличная поддержка LiveData и Flow;
  \item Интеграция с архитектурными компонентами Android;
  \item Улучшенная совместимость с Kotlin.
\end{itemize}